\documentclass{beamer}
\usepackage{minted}
\usetheme{metropolis}
\title{Flask web evaluation for QtRvSim}
% March 16th 2024
\date{16.3.2024}
\author{Jakub Pelc}
\institute{Faculty of Electrical Engineering, Czech Technical University in Prague}
\begin{document}
	\maketitle
	\section{A little bit about Flask}
	
	\begin{frame}{Flask}
		Flask is a micro web framework written in Python. It provides a simple way to create web applications. \par

		As opposed to Django, Flask is not an all-inclusive framework. It is designed to be simple and easy to use. \par

		To start creating a web application, the only thing you need to do is to install Flask and start writing your Python code.

		To run the local webserver, the command \texttt{flask run} is used.

	\end{frame}

	\begin{frame}[fragile]
		\begin{minted}{python}
	from flask import Flask

	app = Flask(__name__)

	@app.route("/")
	def hello():
		return "<p>Hello, World!</p>"
		\end{minted}
	\end{frame}

	\begin{frame}{Loading template files}
		We can utilize Flask to load HTML templates, instead of writing all of the HTML code in the \texttt{app.py} file. \par
		
		We can also pass variables to the template, and detect which HTTP method was used to access the page.
	\end{frame}


	\begin{frame}[fragile]
		\small
		\inputminted{python}{examples/2/app.py}
	\end{frame}

	\begin{frame}{Jinja2 templating}
		This is still not ideal, now we need to create a complete HTML page for every route. \par

		We will now use the Jinja2 templating engine to create a base template, and then extend it for every route.

		We can also add static files, such as CSS stylesheets, or images. \par
	\end{frame}

	\begin{frame}[fragile]
		\small
		\inputminted{html}{examples/3/templates/base.html}
	\end{frame}

	\begin{frame}[fragile]
		\small
		\inputminted{html}{examples/3/templates/register.html}
	\end{frame}

	

\end{document}